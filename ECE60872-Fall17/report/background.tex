\section{Background}
\label{sec:background}

In this section, we present background on edge computing and the simulator
we used in our later evaluation.

\subsection{Edge Computing}
Cloud computing and mobile computing are now used everywhere.
After several decades of sustained effort by many researchers, both of them have
solid core concepts, techniques and mechanisms. Many applications running on mobile devices nowadays
leverage cloud computing to overcome the resource limits of mobile devices. However,
WAN delays in the critical path of user interaction can hurt the application
performance and responsiveness~\cite{cloudlets09}.

Edge computing is a new computing paradigm which moves the computation closer to the mobile devices.
The reousrce in the proximity used in edge computing can be a network resource or a computational resource
in between mobile devices and the cloud. Satyanarayanan et al.~\cite{cloudlets09} introduced the concept
of cloudlet, which is a trusted, resource-rich computer or cluster of computers that's well-connected to
the Internet and available for use by nearby mobile devices. Cloudlet has been extensively discussed in the
literatures~\cite{edge-computing, Cloudlets12,hu-apsys16,ChaufournierSLN17}.

An edge computing infrastructure can be composed of many edge servers and management system softwares.
The management system needs to decide where to offload a task, and how to ensure reliability and availability
in case of failures. Recent work \cite{hu-apsys16,COMET} show that application partition schema, and network latency
both have significant impacts on edge computing performance.

\subsection{EdgeCloudSim}
TODO
