\section{Introduction}
The traditional cloud computing infrastructure has a two-level
architecture: centralized cloud platform and end-user mobile devices.
Normally, compute-intensive capabilities such as speech recognition,
computer vision, machine learning algorithms, and augmented reality
are pushed to cloud, only the final result is sent back to mobile
devices and presented to users.

One of the critical challenges in cloud-backed mobile computing
is the end-to-end network responsiveness between the mobile device
and associated cloud. When the use of cloud resources is in the
critical path of user interaction, the long network latency is very
likely to annoy a mobile user.

To mitigate this issue, there is an emerging trend to offload
such computation to edge devices in close proximity to mobile devices.
Edge computing~\cite{url:openedgecomputing,edge-computing} is a new paradigm in which a lot computing and storage
resources are pushed to the edge of Internet. Edge computing offers
lower latency and a reduction in data traffic through cloud services.
The edge computing introduces another layer into this architecture, and it acts as the
middle layer of the new three-tier hierarchy: mobile, edge devices, and cloud.

However, decentralized edge devices have some intrinsic issues.
For example, normally edge devices are not maintained as well
as centralized data center servers. To make it worse, edge devices
may be exposed to public areas, which increases the probability
of edge device failures. Edge computing based system must take this
into account: it must maximize the benefit provided by edge computing while
minimize the impact of various failures.

To confront these challenges, we propose {\em resilient edge computing}, a robust framwork that provides
low-latency, high-availability and high-reliability computation.
Resiliency~\cite{Resiliency-survey} is defined as the capacity of
a system or an infrastructure to remain reliable, failure tolerant, and dependable
in case of any failures that result in a temporal or permanent service disruption.
Resiliency in cloud computing can classified into two major groups: resiliency of
infrastructure, and resiliency of applications. In this work, we mainly focus
on resiliency of applications. Because the main goal of edge computing is to improve
the usability of mobile applications.

To have a better understanding of resilient edge computing, we fisrt use EdgeCloudSim~\cite{edgecloudsim},
an edge computing simulator based on CloudSim~\cite{cloudsim},
to evaluate the basic performance improvement provided by edge computing and how edge device failures can impact
the framework performance. We then implement an image processing application which is
deployed in all three tiers. We show that real-world applications can benefit from edge computing
despite failures. We also propose and evaluate some runtime strategies to handle different
failure scenarios in edge computing.

The rest of the paper is organized as follows. Section 2 presents the background about
edge computing and the simulator we used in our evaluation. Section 3 presents our simulation results.
We discuss resilient edge computing, along with several policies and mechanisms in section 4.
Section 5 presents implementation details and evaluation results of our image-processing application
We cover related work in section 6 and conclude in section 7.
