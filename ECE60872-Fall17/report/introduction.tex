\section{Motivation}
The traditional cloud computing infrastructure has a two-level
architecture: centralized cloud platform and end-user mobile devices.
Normally, compute-intensive capabilities such as speech recognition,
computer vision, machine learning algorithms, and augmented reality
are pushed to cloud, only the final result is sent back to mobile
devices and presented to users.

One of the critical challenges in cloud-backed mobile computing
is the end-to-end network responsiveness between the mobile device
and associated cloud. When the use of cloud resources is in the
critical path of user interaction, the long network latency is very
likely to annoy a mobile user.

To mitigate this issue, there is an emerging trend to offload
such computation to edge devices in close proximity to mobile devices.
Edge computing~\cite{url:openedgecomputing,edge-computing} is a new paradigm in which a lot computing and storage
resources are pushed to the edge of Internet. Edge computing offers
lower latency and a reduction in data traffic through cloud services.
The edge computing introduces another layer into this architecture, and it acts as the
middle layer of the new 3-tier hierarchy: mobile devices, edge devices,
and cloud servers.

However, decentralized edge devices also introduced many issues.
For example, normally edge devices are not maintained as well
as centralized data center servers. To make it worse, edge devices
may be exposed to public areas, which increases the probability
of device failure. In all, the possible low reliability of edge
devices will have implications in the 3-tier hierarchy.

To confront these challenges, in this work I want to explore
how edge device failures can impact the framework performance,
and how to mask those failures by making edge computing more resilient,
such that applications using this framework will have the lowest performance decrease.

\section{Resilient Edge Computing}
Resiliency~\cite{Resiliency-survey} is defined as the capacity of
a system or an infrastructure to remain reliable, failure tolerant, and dependable
in case of any failures that result in a temporal or permanent service disruption.
Resiliency in cloud computing can classified into two major groups: 1) resiliency in
infrastructure and (2) resiliency in applications. In this work, I will only focus
on resiliency in applications.

Resilient edge computing means that 3-tier hierarchy can sustain any failures, and
provide reliability, availability guarantees for both applications and end users
regardless any failures in any edge devices.

In this work, I assume the following possible failure models and different combinations
of them:
\begin{enumerate}
\item Edge device crash
\item Network failure between mobile and edge device
\item Network failure between cloud and edge device
\end{enumerate}

For the first case, I assume the edge device is crashed, either due to software
corruption or hardware failure. The edge device will not be able to handle, or
route any requests from both cloud and mobile devices. For the last two cases,
I assume network is unreachable by both ends. However, I assume that there is
still one available connection, either from edge device to cloud, or to mobile.
Otherwise, there is no way to recover.

Despite all these failures in the system, the goal of resilient edge computing
is that any applications that are using edge computing service can continue
running, with slight decreased performance.

\section{Plan}
In the first stage of this project, I plan to use cloud simulator CloudSim~\cite{cloudsim}
to simulate both cloud servers and edge components (or cloudlet). CloudSim
is able to simulate both device crash and network failure, which satifies my
failure models. As for the applications, I found several interesting ones from ~\cite{quantify-edge}.
For example, there is a face recognition application FACE and an augmented reality application MAR.
I will choose to port one application first.

Combined together, the goal of my first stage is to: 1) able to run ported applications
on CloudSim without any injected faults, 2) able to handle injected faults as mentioned
above without hurting running applications. To handle the injected faults, I will design
policies at all three ends and implement mechanisms to detect and recover.

In the next stage, I will run on real cloud servers and edge devices (e.g., tablet or desktop).
By manually shutdown machines or cut out network, I will be able to verify the correctness
of my policy and mechanism designed for the system. Along building the system, I will read
more on edge computing side and find more possible optimization possibilities for edge computing.

\section{Previous Work}
Hu et.al~\cite{quantify-edge} has evaluated the impact of edge computing on mobile applications.
Ha et.al~\cite{Ha2015OpenStackFC} developed OpenStack++ as an extension to OpenStack framwork,
for cloudlet development. ~\cite{edge-computing} suggested that edge devices can be used
to mask cloud outages. However, how edge devices failure can be handled properly is not well studied.
