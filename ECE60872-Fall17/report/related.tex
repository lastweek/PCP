\section{Related Work}
\label{sec:related}

Edge computing is an emerging area. Its decentralized approach resembles
the rise of personal computing in 1980s.
There have been a host of literatures~\cite{edge-computing,edgecloudsim,cloudlets09,Cloudlets12,hu-apsys16,ChaufournierSLN17}
on edge computing in the past few years.
Moreover, industry investments in edge computing have grown dramatically recently.
The Open Edge Computing\cite{url:openedgecomputing} initiative was launched in 2015,
and the Open Fog Consortium~\cite{url:openfog} was created in 2015. Both of them have major IT vendors involved.

Different from application-specific edge device runtime,
Satyanarayanan et al.~\cite{cloudlets09} proposed the concept of VM-based cloudlets. A cloudlet is a trusted,
resource-rich computer that's well-connected to the Internet and available for use by nearby mobile devices.
Cloudlet resembles the idea of "Data center in a box", but in close proximity to mobile devices.
However, VM live migration in edge clouds poses a number of challenges. Migrating VMs between geographically
separate locations over slow wide-area network results in large migration times and high unavailbility of the application.

To solve this challenge, Chaufournier et al.~\cite{ChaufournierSLN17} proposed to use MPTCP to speed up VM migration, which can reduce
migratiuon time by up to 50\%.
Hu et al.~\cite{hu-apsys16} explored the impact of partition schemas, offloading sites on edge computing performance.
They found that the choice of offloading sites is important for both prepartitioned applications and
for dynamic offloading frameworks such as COMET~\cite{COMET}.
